%Custom abstract
\singlespacing
\selectlanguage{english}
\section*{\centering{\footnotesize{ABSTRACT}}}
    % \begin{adjustwidth}{18pt}{38pt}
    %         \setstretch{1.0}{
    %         \footnotesize

The Santa Angélica intrusive complex (SAIC) in southeastern Brazil exhibits a distinctive reversely zoned pattern, featuring two lobes with concentric bull's-eye shape. Each lobe consists of a mafic core surrounded by a magma mixture zone with granitic borders. The intrusion took place during the post-collisional (Cambrian) phase of the Araçuaí orogen, providing valuable insights into the orogen's evolution. To unravel crucial information from post-collisional bodies like SAIC, gravity modeling is employed. This methodology requires meticulous data processing to extract residual gravimetric anomalies associated with the intrusion. Our objective is to enhance the open-access gravimetric survey of SAIC through reprocessing, aiming to derive the gravimetric disturbance. This is achieved by applying equivalent layer inversion to predict data within a regular grid area. The results were coherent with the SAIC's geology with positive peaks on the mafic cores, positive values over the mixture zone, and negative values for the granitic borders. Further proving the reliability of the applied technique.
            
            
        % }
        % \end{adjustwidth}


% \newpage
% \selectlanguage{brazil}
% \section*{\centering{\footnotesize{RESUMO}}}
% 	\begin{adjustwidth}{18pt}{38pt}
% 	    \setstretch{1.0}{
%      \footnotesize
%      Os dados paleomagnéticos são geralmente obtidos a partir de amostras cilíndricas completas, onde o sinal resulta da soma dos momentos magnéticos de centenas de milhares a milhões de partículas magnéticas dentro do volume da amostra. Isso normalmente inclui portadores de remanescência estáveis e instáveis. Recentemente, as técnicas de microscopia magnética permitiram a investigação de grãos individuais através da imagem direta de seus campos magnéticos. No entanto, a determinação dos momentos magnéticos de grãos individuais é dificultada pela ambiguidade intrínseca na inversão dos dados do campo potencial, bem como pelo grande número de grãos encontrados em qualquer imagem de microscopia. Apresentamos um algoritmo rápido e semi-automatizado capaz de estimar a posição e a magnetização de cada fonte ferromagnética (l.s) usando apenas os dados de microscopia magnética. Nosso algoritmo funciona em três etapas: (i) aplicamos técnicas de processamento de imagem para identificar e isolar as fronteiras das janelas de dados para cada fonte; (ii) com essas fronteiras de janelas, a posição das fontes é estimada usando a deconvolução de Euler; e, finalmente, (iii) usando as informações de posição, o algoritmo é capaz de estimar a direção e intensidade do momento dipolar magnético para cada fonte através de um problema inverso linear sobredeterminado usando uma aproximação dipolar. O método não requer nenhum tipo de informação adicional sobre a amostra ou as fontes. Foram realizados testes de sensibilidade para estimar a estabilidade de nossa rotina em relação à profundidade das partículas, razão sinal-ruído e não dipolaridade das fontes. Testes com dados sintéticos simples mostram a alta efetividade da metodologia na recuperação das informações de posição e magnetismo tanto para fontes dipolares quanto não dipolares. Dados sintéticos mais complexos, incluindo mais de 100 partículas magnéticas diferentes, foram criados para emular dados reais de rochas. Os resultados obtidos com esses dados também demonstram a viabilidade e robustez do algoritmo para estimar semi-automaticamente a posição e o momento magnético de um grande número de partículas. Isso é confirmado por meio de uma aplicação a dados reais em que conseguimos recuperar as direções esperadas de remanescência isotérmica bimodal que foram induzidas na amostra. Dada a sua natureza semi-automática, baixo custo de processamento e possibilidade de inversão simultânea do momento magnético de um grande número de partículas magnéticas, a metodologia aqui proposta representa um avanço na capacidade de aplicar a microscopia magnética em estudos paleomagnéticos.
%         }
% 	 \end{adjustwidth}
\selectlanguage{english}
\doublespacing
% \newpage