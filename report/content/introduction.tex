\section{Introduction}\label{sec:intro}

Several branches of Earth Sciences have demonstrated the importance of spatial resolution on a microscopic scale. For example, geochemistry and
geochronology applications have benefited from point-wise analyses
and compositional maps, allowing significant advances in the understanding of
igneous, metamorphic and sedimentary processes \citep{Verberne2020, Barnes2019,
Davidson2007}. Classical paleomagnetic techniques, on the other hand, consist
of analyzing bulk samples, where the magnetic signal of a single specimen is
the result of the sum of moments of a large assembly of ferromagnetic grains
\citep{Dunlop1997}. Typically, a standard paleomagnetic sample of
approximately \qty{10}{\cm\cubed} would contain hundreds of thousands
to millions of magnetic particles with sizes varying from magnetically stable
single-domain (SD) and vortex state grains (also called pseudo-single domain,
PSD) with sizes below \qty{1}{\um}, to large ($\gg \qty{1}{\um}$) grains with
multi-domain (MD) magnetic structures, which are less stable magnetic recorders
\citep{Berndt2016}. These large MD grains usually conceal the signal of the SD
and PSD grains, and techniques of step-wise thermal and magnetic treatments are
needed to unveil this more stable and reliable magnetic record
\citep{Tauxe2018}. Recently, magnetic microscopy techniques opened the
possibility of obtaining magnetic field maps at the micro-scale and recovering the
magnetization of each grain, therefore enabling the separate analysis of stable
and unstable magnetic particles \citep{DeGroot2018, Lima2014, Weiss2007,
DeGroot2014}.

In order to apply magnetic microscopy to paleomagnetic studies, it is necessary
to recover from the magnetic images a large number of individual magnetic
moments, corresponding to at least tens of thousands of stable fine-grained
grains ($< \qty{1}{\um}$), in order to provide statistical significance to the
remanence vector \citep[e.g., ][]{Berndt2016}. Nowadays, with the development of
magnetic microscopy techniques, this task is no longer limited by the
resolution of magnetic microscopes \citep{Fu2020, Weiss2007, DeGroot2018,
Glenn2017, Lima2014}, but essentially by the intrinsic problem presented by the
ambiguity in the inversion of potential field data \citep{Barbosa2011,
DeGroot2021, Oliveira2015Estimation}, and ultimately by the lack of a fast and
automated way to recover such a large number of individual magnetic moments
from a set of magnetic images \citep{CortesOrtuno2022, Lima2013, Lima2009}. A
solution to the non-uniqueness of magnetic moment inversion is to add
independent prior information, such as the position of the ferromagnetic
particles \citep{Fabian2019}. This can be obtained, for example, from X-ray
computed tomography \citep[microCT; ][]{Fabian2019, DeGroot2021, DeGroot2018}.
Nonetheless, the standard microCT techniques do not provide adequate resolution
to resolve the finer and more stable magnetic grains \citep{CortesOrtuno2022,
DeGroot2021}, whereas other more sophisticated techniques such as ptychographic
X-ray tomography \citep[e.g., ][]{Maldanis2020} are not readily available and
too time-consuming to be routinely used in paleomagnetic studies.

Another route to be explored in the inversion of magnetic microscopy images is
to obtain all the information, i.e. the magnetic moment and the position of the
sources, from the magnetic data itself \citep[e.g., ][]{Fu2020}. For that, we can
explore the techniques developed in exploration geophysics, in spite of the
differences between aeromagnetic surveys and magnetic microscopy
\citep{Lima2013}. Magnetic microscopy images commonly show the combined signal
of multiple magnetic particles and can vary greatly in wavelength, strength,
and spatial separation, depending on the natural remanent magnetization (NRM) and location of each particle. We
usually assume that the signal measured by the magnetic microscope is the
vertical component of the magnetic induction vector ($b_z$), the measurements
are performed on a regular grid with evenly spaced grid points and at a
constant height, and the data are contaminated with pseudorandom Gaussian noise and
long-wavelength noise (akin to a regional signal in aeromagnetic data). Here,
we provide a methodological routine to retrieve the individual magnetic moment
of ferromagnetic grains in magnetic microscopy images following the approach
devised by \citet{Oliveira2015Estimation} for the interpretation of
aeromagnetic anomalies. The method we propose allows one to quickly and
semi-automatically estimate the individual magnetic moment vector of the stable
magnetic carriers, making use of only the magnetic images themselves and an
assumption of approximately dipolar sources. If used on a large scale, the
method provides the means to scan large areas of the rock sample, attaining
potentially the number of magnetic moments necessary for paleomagnetic studies.


